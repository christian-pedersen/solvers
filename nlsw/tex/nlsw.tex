\documentclass[11pt]{report}
\usepackage[utf8]{inputenc}
\usepackage{amsmath}


\begin{document}

\chapter*{The non-linear shallow water equations}

\section*{Mathematical formulation}

The scaled non-linear shallow water equations is a set of partial 
differential equations 
\begin{equation}
\frac{\partial \mathbf{u}}{\partial t} + \mathbf{u}\cdot\nabla\mathbf{u} = -\nabla \eta
\label{eq:N2}
\end{equation}
\begin{equation}
\frac{\partial \eta}{\partial t} = -\nabla\cdot\left[\mathbf{u}\left(\eta + h\right)\right]
\label{eq:continuity}
\end{equation}

\noindent where $\mathbf{u}$ is the wave velocity, $\eta$ is the surface elevation
and $h$ is the water depth. Eq.\ref{eq:N2} is the equation of motion derived from Newtons
second law of motion and Eq.\ref{eq:continuity} is the continuity equation.


\section*{Finite element formulation}

By implementing the Galerkin method, i.e. multiplying each equation with a test function
and performing integration by parts (IBP), we can obtain a finite element formulation;
Find $u_h\in H^1$ and $\eta _h\in L^2$ such that
\begin{equation}
a(u_h, v) + b(\eta_h, v) = 0 \qquad \forall v \in H^1
\end{equation}
\begin{equation}
c(u_h, q) + d(\eta_h, q) = 0 \qquad \forall q \in L^2
\end{equation}
\noindent where
\begin{equation}
\begin{split}
a(u_h, v) &= \left(\frac{\partial u_h}{\partial t}, v \right)_\Omega + \left(u_h\cdot\nabla u_h, v \right)_\Omega \\
b(\eta _h, v) &= \left(\eta _h, v\right) \\
c(u_h, q) &=  \\
d(\eta_h, q) &=
\end{split}
\end{equation}


\section*{Temporal discretization}



\end{document}
