\documentclass[11pt]{article}
\usepackage[utf8]{inputenc}
\usepackage{amsmath}


\title{Elasticity \\ an introduction}
\author{Christian Pedersen}

\begin{document}

\maketitle


\section*{Linear elastostatics}

Linear elastostatics can be described by the governing equations
%
\begin{subequations}
\begin{equation}
f_b + \nabla\cdot \sigma  = 0
\end{equation}
\begin{equation}
\sigma = 2\mu\epsilon(u) + \lambda tr(\epsilon(u))\delta 
\end{equation}
\begin{equation}
\epsilon(u) = \frac{1}{2}\left(\nabla u + \nabla u^T \right)
\end{equation}
\end{subequations}
%
The first equation is Newton's second law of mechanical equilibrium where
$f_b$ is the body force, i.e. gravity, and $\nabla\cdot\sigma$ is the 
contact force. The stress tensor $\sigma$ is derived from Hooke's law 
where we have a linear relation between stress and strain $u$ and
 $\epsilon(u)$ is Cauchy's strain tensor.
By combining the equation set $(1)$ we obtain
%
\begin{equation}
f_b + \mu\nabla^2u + (\lambda + \mu)\nabla\nabla\cdot u = 0
\end{equation}
%
which is called the \textit{Navier-Cauchy equilibrium equation}.

\subsection*{Finite element formulation}

u is now expressed as a sum of piecewise linear basis functions
$u=\sum_j c_j\psi_j$. After we multiply equation $(2)$ with a test function 
$v$ and perform integration by parts, a FEM formulation is; 

Find $u\in H^1$ such that 
%
\begin{equation}
a(u,v) = L(v) \;, \quad \forall v \in H^1
\end{equation}
%
where
% 
\begin{equation}
\begin{split}
a(u, v) &= \mu(\nabla u, \nabla v) + (\mu + \lambda)(\nabla \cdot u, \nabla \cdot v)\\
L(v) &= (f, v)
\end{split}
\end{equation}
%



\section*{Linear elastodynamics}

Linear elastodynamics can be described as linear elastostatics with an additional inertia term
\begin{equation}
f_b + \mu\nabla^2u + (\lambda + \mu)\nabla\nabla\cdot u = \rho\frac{\partial^2 u}{\partial t^2}
\end{equation}



\end{document}


